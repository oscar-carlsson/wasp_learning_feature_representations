\documentclass[a4paper]{article}

\usepackage[english]{babel}
\usepackage[utf8]{inputenc}
\usepackage{amsmath, amsthm}
\usepackage{graphicx}
\usepackage[colorinlistoftodos]{todonotes}
\usetikzlibrary{trees}
\usepackage{geometry}
\usepackage{amssymb}
\usepackage{enumitem}
\usepackage{fancyhdr}
\usepackage{tikz}
\usetikzlibrary{trees}
\pagestyle{fancy}

\usepackage{comment}
\usepackage{float}
\usepackage{mathrsfs}
\usepackage{physics}
\usepackage{bbm}
\usepackage[hidelinks]{hyperref}
\usepackage{parskip}
\usepackage{lipsum}

\theoremstyle{definition}
\newtheorem{definition}{Definition}
\newtheorem{example}{Example}
\newtheorem{remark}{Remark}

\theoremstyle{plain}
\newtheorem{lemma}{Lemma}
\newtheorem{proposition}{Proposition}
\newtheorem{theorem}{Theorem}
\newtheorem{corollary}{Corollary}

\renewcommand{\headrulewidth}{0pt}
\renewcommand{\footrulewidth}{0pt}

\newcommand{\approptoinn}[2]{\mathrel{\vcenter{
  \offinterlineskip\halign{\hfil$##$\cr
    #1\propto\cr\noalign{\kern2pt}#1\sim\cr\noalign{\kern-2pt}}}}}

\newcommand{\appropto}{\mathpalette\approptoinn\relax}

\begin{document}


{\center\Large\scshape Learning Feature Representations\par}
{\center\large\scshape Module 1 Homework\par}
\vspace{2mm}
{\center\scshape Oscar Carlsson\\Jimmy Aronsson\par}
\vspace{1mm}
{\center\small\scshape \today\par}
\vspace{7mm}
%\maketitle

In this document, we summarize our work on the first Homework.

\section*{\center Exercise 1}

In the first exercise, we fit a multivariate Gaussian to MNIST patches'

\subsection*{NCE}

\lipsum[2] 

\begin{equation}\label{J_function}
J(\theta) \propto \mathbb{E}_{x \sim p_d} \left[ \log \frac{p_\theta(x)}{p_\theta(x) + \nu p_n(x)}\right] + \nu \mathbb{E}_{x \sim p_n} \left[ \log \frac{ \nu p_n(x)}{p_\theta(x) + \nu p_n(x)}\right].
\end{equation}


We approximate the right-hand side of equation \eqref{J_function} using the empirical estimate
$$\frac{1}{N} \sum_{i=1}^N \log \frac{p_\theta(x_i)}{p_\theta(x_i) + \nu p_n(x_i)} + \frac{\nu}{M} \sum_{j=1}^M \log\frac{\nu p_n(x_j')}{p_\theta(x_j') + \nu p_n(x_j')},$$
where $x_i \sim p_d$ are training examples and $x_j' \sim p_n$ are drawn from the noise distribution $\mathcal{N}(\mathbf{0},\Sigma_n)$.  We have further chosen the model distribution $\mathcal{N}(\mathbf{0},\Sigma_\theta)$, and the above expression can thus be simplified quite a bit. Start by rewriting both terms in the following way:
\begin{alignat*}{1}
\log \frac{p_\theta(x)}{p_\theta(x) + \nu p_n(x)} &= -\log\left( 1 + \nu \frac{p_n(x)}{p_\theta(x)}\right),\\
\log \frac{\nu p_n(x)}{p_\theta(x) + \nu p_n(x)} &= - \log \left( \frac{p_\theta(x)}{p_n(x)} + \nu\right),
\end{alignat*}
and then insert the relative probability
$$w(x) = \frac{p_n(x)}{p_\theta(x)}  = \sqrt{\frac{|\Lambda_n|}{|\Lambda_\theta|}} \exp\left( -\frac{1}{2} x^T \left( \Lambda_n - \Lambda_\theta\right) x\right),$$
to obtain the relatively simple expression
$$\boxed{J(\theta) \appropto -\frac{1}{N} \sum_{i=1}^N \log\left( \nu w(x_i) + 1\right) -\frac{\nu}{M} \sum_{j=1}^M \log\left(w(x_j')^{-1} + \nu\right) }$$

$$\textit{*Insert figures of samples, loss, etc*}$$

\subsection*{Score Matching}


\begin{alignat*}{1}
\nabla_x \log p_\theta(x) &= -\Lambda (x-\mu)\\
\Delta \log p_\theta(x) &= -\trace(\Lambda)
\end{alignat*}







\section*{\center Exercise 2}

The first step was to extract 50,000 image patches of resolution $28\times 28$. We solved this problem by running the following loop: In each iteration, an image from the \texttt{Flickr30k} dataset is loaded, converted to grayscale, and split into multiple patches using the method \texttt{tf.image.extract\_patches}. Two such patches are selected at random and saved, before moving on to the next iteration, and the program terminates after saving 50,000 patches. See \texttt{create\_image\_patches.py} for details.

Next, we computed a constrained Gaussian representing the above data \lipsum[2]































\end{document}