\documentclass[a4paper]{article}

\usepackage[english]{babel}
\usepackage[utf8]{inputenc}
\usepackage{amsmath, amsthm}
\usepackage{graphicx}
\usepackage[colorinlistoftodos]{todonotes}
\usetikzlibrary{trees}
\usepackage{geometry}
\usepackage{amssymb}
\usepackage{enumitem}
\usepackage{fancyhdr}
\usepackage{tikz}
\usetikzlibrary{trees}
\pagestyle{fancy}

\usepackage{comment}
\usepackage{float}
\usepackage{mathrsfs}
\usepackage{physics}
\usepackage{bbm}
\usepackage[hidelinks]{hyperref}
\usepackage{parskip}
\usepackage{lipsum}
\usepackage{pgfplots}

\theoremstyle{definition}
\newtheorem{definition}{Definition}
\newtheorem{example}{Example}
\newtheorem{remark}{Remark}

\theoremstyle{plain}
\newtheorem{lemma}{Lemma}
\newtheorem{proposition}{Proposition}
\newtheorem{theorem}{Theorem}
\newtheorem{corollary}{Corollary}

\renewcommand{\headrulewidth}{0pt}
\renewcommand{\footrulewidth}{0pt}

\newcommand{\approptoinn}[2]{\mathrel{\vcenter{
  \offinterlineskip\halign{\hfil$##$\cr
    #1\propto\cr\noalign{\kern2pt}#1\sim\cr\noalign{\kern-2pt}}}}}

\newcommand{\appropto}{\mathpalette\approptoinn\relax}

\begin{document}


{\center\Large\scshape Learning Feature Representations\par}
{\center\large\scshape Module 1 Homework\par}
\vspace{2mm}
{\center\scshape Oscar Carlsson\\Jimmy Aronsson\par}
\vspace{1mm}
{\center\small\scshape \today\par}
\vspace{7mm}
%\maketitle

In this document, we summarize our work on the first Homework.

\section*{\center Exercise 1}

In this first exercise, we attempt to model the (unknown) distribution of MNIST images, $x \sim p_d$, and hopefully generate synthetic images that look realistic. More precisely, we remove the empirical mean $\mu$ from each training image and see whether the remaining noise $x - \mu$ can be modeled with a multivariate Gaussian distribution $\mathcal{N}(\mathbf{0},\Sigma_\theta)$ with learned precision matrix $\Lambda_\theta = \Sigma_\theta^{-1}$. Synthetic images could then be created by either sampling noise from the model distribution and adding back the empirical mean, 
$$x' = \mu + \epsilon, \qquad \epsilon \sim \mathcal{N}(\mathbf{0},\Sigma_\theta),$$
or by using the precision matrix directly:
$$x' = \mu + A_\theta \epsilon, \qquad A_\theta = \Lambda_\theta^{-1/2}, \epsilon \sim \mathcal{N}(0,1/100).$$

Two different methods have been used to estimate the precision matrix:
\begin{itemize}
\item Noise-contrastive estimation (NCE), and
\item Score matching (SM).
\end{itemize}
Seeing as the underlying ideas and general analysis of these methods have already been discussed by Christopher Zach in his \texttt{presentation\_part1}, we only discuss those additional details which are of relevance to us.

\subsection*{NCE}

As explained in the presentation by Zach, noise-contrastive estimation (NCE) casts the estimation of distribution parameters as a supervised learning problem. This effectively means teaching the model distribution $p_\theta$ to distinguish between real data $x \sim p_d$ and noise data $x' \sim p_n$, where the noise distribution should be similar enough to the data distribution for this classification problem to be challenging; we want the model distribution $p_\theta$ to learn  the most essential properties of $p_d$.

After choosing a suitable noise distribution $p_n$, we construct a data set $\{(z_i,x_i)\}$ by flipping a weighted coin $z \sim \text{Bern}(\eta)$ multiple times, $z_1,z_2,z_3,\ldots$, and letting each result $z_i \in \{0,1\}$ decide whether to sample $x_i$ from the data distribution ($z_i = 1$) or from the noise distribution ($z_i = 0$).

\begin{equation}\label{J_function}
J(\theta) \propto \mathbb{E}_{x \sim p_d} \left[ \log \frac{p_\theta(x)}{p_\theta(x) + \nu p_n(x)}\right] + \nu \mathbb{E}_{x \sim p_n} \left[ \log \frac{ \nu p_n(x)}{p_\theta(x) + \nu p_n(x)}\right].
\end{equation}


We approximate the right-hand side of equation \eqref{J_function} using the empirical estimate
$$\frac{1}{N} \sum_{i=1}^N \log \frac{p_\theta(x_i)}{p_\theta(x_i) + \nu p_n(x_i)} + \frac{\nu}{M} \sum_{j=1}^M \log\frac{\nu p_n(x_j')}{p_\theta(x_j') + \nu p_n(x_j')},$$
where $x_i \sim p_d$ are training examples and $x_j' \sim p_n$ are drawn from the noise distribution. Next, we simplify the above expression above by rewriting both terms in the following way:
\begin{alignat*}{1}
\log \frac{p_\theta(x)}{p_\theta(x) + \nu p_n(x)} &= -\log\left( 1 + \nu \frac{p_n(x)}{p_\theta(x)}\right),\\
\log \frac{\nu p_n(x)}{p_\theta(x) + \nu p_n(x)} &= - \log \left( 1 + \frac{1}{\nu} \frac{p_\theta(x)}{p_n(x)} \right),
\end{alignat*}
and then insert the relative probability
$$w(x) = \frac{p_n(x)}{p_\theta(x)}  = \sqrt{\frac{|\Lambda_n|}{|\Lambda_\theta|}} \exp\left( -\frac{1}{2} x^T \left( \Lambda_n - \Lambda_\theta\right) x\right),$$
to obtain the relatively simple expression
\begin{equation}\label{J_function2}
J(\theta) \appropto -\frac{1}{N} \sum_{i=1}^N \log\left( \nu w(x_i) + 1\right) -\frac{\nu}{M} \sum_{j=1}^M \log\left(\frac {1}{\nu w(x_j')} + 1\right).
\end{equation}
We found that $w(x)$ is typically very small in practice, hence the sum $(\nu w)^{-1} + 1$ is dominated by its first term. Its logarithm can thus be approximated by the numerically more stable expression
\begin{alignat*}{1}
\log(\frac{1}{\nu w(x)} + 1) \approx -\log \nu w(x) = \frac{1}{2}x^T (\Lambda_n - \Lambda_\theta) x - \frac{1}{2} \log \left(\nu^2  \frac{|\Lambda_n|}{|\Lambda_\theta|}\right).
\end{alignat*}
It would also be possible to remove the first sum in equation \eqref{J_function2}, since $\log(\nu w(x) + 1) \approx \log 1$. We decided to keep it, however, because it didn't cause computational problems and we didn't want our estimate to be independent of the real training data. Thus, our final estimate is
$$\boxed{J(\theta) \appropto -\frac{\nu}{2}\log \left(\nu^2 \frac{|\Lambda_n|}{|\Lambda_\theta|}\right) -\frac{1}{N} \sum_{i=1}^N \log\left( \nu w(x_i) + 1\right) + \frac{\nu}{2M} \sum_{j=1}^M {x_j'}^T (\Lambda_n - \Lambda_\theta) x_j'}$$
We also obtained an expression for the gradient $\nabla J(\theta)$ in terms of the precision matrix $\Lambda_\theta$, though we found this expression rather bulky and difficult to handle. Instead, we used \texttt{tf.GradientTape} to compute the gradient and update $\Lambda_\theta$.  Two approaches were considered for keeping $\Lambda_\theta$ positive definite and retaining its sparse 4-/8-connected neighbourhood-structure after each epoch:
\begin{enumerate}
\item Writing the precision matrix as $\Lambda_\theta = (A_\theta ^T A_\theta) \cdot M$ for a learned matrix $A_\theta$ and a pre\-defined masking matrix $M \in \{0,1\}^{28\times 28}$ that is applied element-wise, enforcing the neighbourhood structure by killing undesired matrix elements.

The matrix product $A_\theta^T A_\theta$ is guaranteed to be symmetric positive definite whenever $A_\theta$ is invertible, which any square matrix almost surely is. Combined with the fact that element-wise products of positive definite matrices is again positive definite, we hoped this would prove that $\Lambda_\theta$ is symmetric positive definite. Unfortunately, we eventually realized that our masking matrix is not positive definite, so we cannot guarantee that $\Lambda_\theta$ is, either. Learning $A_\theta$ also turned out to be slower than the approach below.

\item Forcing a symmetric gradient by throwing away its lower triangular part and replacing it with the transpose of its upper triangular part. We then applied the previously mentioned masking matrix $M$ to force the neighbourhood structure on the gradient. This ensures that $\Lambda_\theta$ is symmetric and retains its neighbourhood structure for all epochs. On the other hand, we still cannot guarantee that $\Lambda_\theta$ remains positive definite.
\end{enumerate}
We ended up choosing the latter approach.\footnote{We later realized $\Lambda_\theta$ can also be modeled through its eigendecomposition, which would make it easy to ensure positive (semi)definiteness by flipping the signs of negative eigenvalues. Due to time constraints and different priorities, however, we never tried implementing this approach.}

$$\textit{*Insert figures of samples, loss, etc*}$$

\subsection*{Score Matching}

When given the choice between cNCE and score matching, we figured the latter would be more interesting, it being a fundamentally different approach than NCE. Fortunately, score matching also turned out to be easy to implement because the relevant analysis had already been excellently performed in the presentation. It allowed us to more or less directly implement the loss function
\begin{alignat*}{1}
J(\mu,\Lambda_\theta) &= \int \frac{1}{2} \|\nabla_x \log p_\theta(x)\|^2 + \Delta \log p_\theta(x) \approx \frac{1}{2N} \sum_{i=1}^N \| \Lambda_\theta (x_i - \mu)\|^2 - \tr(\Lambda_\theta),
\end{alignat*}
and start training. Gradients were again computed with \texttt{tf.GradientTape}, and the symmetry and neighbourhood structure was enforced in the same way as for NCE.

\begin{figure}[H]
\centering
\begin{tikzpicture}[scale=0.8]
\begin{axis}[
tick align=outside,
tick pos=left,
x grid style={white!69.0196078431373!black},
xmin=-0.5, xmax=9.5,
xtick style={color=black},
y dir=reverse,
y grid style={white!69.0196078431373!black},
ymin=-0.5, ymax=9.5,
ytick style={color=black}
]
\addplot graphics [includegraphics cmd=\pgfimage,xmin=-0.5, xmax=9.5, ymin=9.5, ymax=-0.5] {test-000.png};
\end{axis}
\end{tikzpicture}
\caption{Test image and caption}
\end{figure}






\section*{\center Exercise 2}

The first step was to extract 50,000 image patches of resolution $28\times 28$. We solved this problem by running the following loop: In each iteration, an image from the \texttt{Flickr30k} dataset is loaded, converted to grayscale, and split into multiple patches using the method \texttt{tf.image.extract\_patches}. Two such patches are selected at random and saved, before moving on to the next iteration, and the program terminates after saving 50,000 patches. See \texttt{create\_image\_patches.py} for details.

Next, we computed a constrained Gaussian representing the above data \lipsum[2]































\end{document}